\documentclass[10pt, a4paper, spanish]{article}
\usepackage[paper=a4paper, left=1.5cm, right=1.5cm, bottom=1.5cm, top=3.5cm]{geometry}
\usepackage[utf8]{inputenc}
\usepackage[spanish,es-nodecimaldot]{babel}
\usepackage{caratula}
\usepackage[pdfencoding=auto, colorlinks=true, linkcolor=blue]{hyperref}
\usepackage[boxruled, longend]{algorithm2e}
\usepackage{wrapfig}
% \usepackage{tikz}
\usepackage[rightcaption]{sidecap}
% \usetikzlibrary{babel}
\usepackage{float}
\graphicspath{ {imagenes/} }

% \tikzset{nodeList/.style={every node/.style={draw, circle}}}
% \tikzset{pathList/.style={every node/.style={midway, fill=white}}}


\newcommand{\ord}{\ensuremath{\operatorname{O}}}
\newcommand{\nat}{\ensuremath{\mathbb{N}}}

\begin{document}
% CARATULA
\materia{Métodos Numéricos}
\submateria{Primer Cuatrimestre de 2018}
\fecha{\today}
\grupo{Los arboles mueren de pie}
\titulo{Trabajo Práctico 2}

\integrante{Giudice, Carlos}{694/15}{carlosr.giudice@gmail.com}
\integrante{Szperling, }{763/15}{juanju@dc.uba.ar}

\maketitle

\newpage
\tableofcontents

\newpage
\section{Introducción}

Este trabajo práctico tiene como objetivo el desarrollo y estudio de una herramienta que permita reconocer rostros en imágenes. El algoritmo capaz de llevar esto a cabo es uno de clasificación supervisado que fue entrenado con un lote de fotografías conocido, de modo que le sea posible reconocer otras fotografías de esos rostros aprendidos que no se encuentren en la base de datos de entrenamiento.

Las imágenes de las que disponemos corresponden a cuarenta y un personas, habiendo diez imágenes diferentes por persona.

Si la imagen que entra como parámetro, es de una persona que no pertenece al grupo con que se entrenó al algorítmo, entonces éste indicará a cual de los cuarenta y un se parece más.

\subsection{Evaluación}

Para el estudio de esta herramienta es necesaria la evaluación de los métodos y la correcta elección de sus parámetros. Una forma de evaluación es la estimación de la correctitud de la clasificación, para lo cual es necesita conocer previamente a qué persona corresponde cada imagen. La forma de realizar esto es particionar la base de entrenamiento en dos, utilizando una parte de ella en forma completa para el entrenamiento y la restante como test, pudiendo así corroborar la clasificación realizada, al contar con el etiquetado del entrenamiento.

Sin embargo, realizar toda la experimentación sobre una única partición de la base podría resultar en una incorrecta estimación de parámetros, por ejemplo, podría dar overfitting. Por lo tanto, se implementó la técnica de \textit{K-fold cross validation} que resulta estadísticamente más robusta.

El resultado del algoritmo final fue medido con distintas métricas (Accuracy, Curvas de precisión/recall).

% ____________A CORREGIR_______________



\newpage
\section{Desarrollo}

\subsection{kNN}

El algorimo kNN (k Nearest Neighbours) se basa en el análisis de un conjunto de puntos del espacio para determinar a qué clase corresponde el nuevo objeto. En este caso cada imagen estará representada como un vector donde cada elemento es un píxel distinto de la misma. El conjunto de puntos elegido serán aquellas $k$ imágenes que más cerca se encuentren de la imagen a clasificar. Se clasificará a la nueva imagen como perteneciente a la clase que mayor representantes tenga en este conjunto de puntos cercanos.

Este algoritmo puede ser sumamente costoso en cuanto al tiempo de cómputo, y si la dimensión de los puntos a clasificar es muy grande, hacer uso del mismo podría resultar impracticable. Es por esto que se implementó un método cuyo objetivo es preprocesar las imágenes para reducir la cantidad de dimensiones de las muestras, permitiendo a kNN trabajar con muestras de una menor cantidad de variables. Este método es conocido como PCA (Principal components analysis).

\subsection{PCA}

El método de análisis de componentes principales se encarga de cambiar de base el conjunto de datos de entrada para obtener una mejor representación de los datos, y además reduce la dimensión de cada elemento tanto como se desee. Al reducir la dimensión de un punto es claro que se pierde información sobre el mismo, pero la particularidad de este método es que, se queda con las componentes más importantes, dejando de lado las que menos información aporten (de allí su nombre). De este modo, la información que descartada es la de menor relevancia, por lo que se lo considera una buena manera de reducir el espacio de la entrada.

Es importante aclarar que el método no solamente reduce la dimensión de los datos, sino que cambia la base de los mismos. Si se utiliza el método para reducir la entrada de kNN, es necesario cambiar a la misma base la imagen a clasificar, de lo contrario estarían comparándose elementos pertenecientes a distintos espacios.

\bigskip

Procedimiento para el cambio de base:

\begin{enumerate}
\item Se define una base de datos de entrenamiento (para $kNN$) como el conjunto $\D = \{x_i : i = 1, . . . , n\}$.
\item Sea $\mu = (x_1 + ... + x_n)/n$ el promedio de las imágenes de $D = \{x_i : i = 1, ..., n\}$ tal que $x_i\in\mathbb{R}^{m}$. Definimos $X\in\mathbb{R}^{nxm}$ como la matriz que contiene en la $i$-ésima fila al vector $(x_i - \mu)^t/  \sqrt{n-1}$. La matriz de covarianza de la muestra $X$ se define como $M = X^t X$
\item Se calcula $X$ y con ella $M$.
\item Se calculan los autovectores de $M$ mediante el método de las potencias, con deflación. Cómo cada iteración en la que calculamos el autovector de la matriz en cuestión, este está asociado al autovalor de máximo módulo, los autovectores que habremos calculado se encontrarán ordenados por relevancia. De esta manera se calculan tan solo $\alpha$ autovectores, con $1 \leq \alpha \leq n$, siendo $\alpha$ la dimensión a la que se quiere reducir las imágenes. 
\item Se contruye la matriz $V$ con los autovectores calculados previamente, dispuestos como columnas. $V$ es la matriz de cambio de base.
\item Por último, se reduce la dimensión de $X$ cambiando su base. El resultado final es $V^tX^t$ que contiene la misma cantidad de imágenes pero expresadas en otra base, y en lugar de tener dimensión $n$, cada una tiene dimensión $\alpha$.
\end{enumerate}\tabularnewline

Es interesante notar que una vez hecho el cambio de base de las imágenes, éstas representan a las imágenes pero si se trata de graficarlas se obtendrá algo muy distinto a lo que era anteriormente y no podrá visualizarse nada en concreto. Solo volviendo a la base original sería posible, pero ya se habrá perdido mucha información por lo que probablemente sea difícil encontrarlo útil.


\subsection{$X^t*X$ y $X*X^t$}
El método de la potencia con deflación se utiliza para obtener tanto los autovectores como los valores asociados de una matríz. Éste método es particualarmente suceptible al tamaño de la matríz que entra como parámetro. Al ser $X\in\mathbb{R}^{nxm}$ (con $m >> n$), la matríz $M$ acarrea un gran costo temporal para el cómputo del método de la potencia con deflación. Por lo que definimos la matríz $\^{M}\in\mathbb{R}^{nxn}$ como $\^{M}=X*X^t$. Como podemos ver, $\^{M}$ depende de la cantidad de imágenes de training set (que tiene como cota superior 410), por lo que será mucho más chica que $M$. Propondremos un método que utiliza los autovectores y autovalores de $\^{M}$ (cuyo cálculo implíca menor costo temporal) para obtener los de $M$.

\\
Primero veamos que relación hay entre los autovectores y los autovalores de {$X^t*X$ y $X*X^t$}:\\
\begin{center}
Sea $X\in\mathbb{R}^{nxm}$, $\^{M}=XX^t$ y $M=X^tX$.\\
Sea $v_i$ el autovector de $\^{M}$ asociado a $\lambda_i$, para  $i = 1, ..., n$.\\
\Rightarrow $\^{M}v_i=\lambda_iv_i$, para  $i = 1, ..., n$. Por definición de autovalor y autovector.\\
\Rightarrow $XX^tv_i=\lambda_iv_i$. Porque $\^{M}=XX^t$. \\
\Rightarrow $X^tXX^tv_i=\lambda_iX^tv_i$. Multiplico a izquierda por $X^t$.\\
 Defino $u_i=X^tv_i$\\
\Rightarrow $X^tXu_i=\lambda_iu_i$\\
Como $M=X^tX$,\\
\Rightarrow$Mu_i = \lambda_iu_i$\\
\therefore $u_i$ es el autovector de $M$ asociado al autovalor $\lambda_i$
\end{center}\\
\bigskip

Procedimiento para obtener los autovectores y autovalores de $M$ a partir de  $\^{M}$ utilizando la definición de $u_i$ que acabamos de formular:
\begin{enumerate}
\item Utilizando el método de la potencia con deflación se calculan los autovectores y autovalores de $\^{M}=X*X^t$ ($v_i$ y $\lambda_i$, para  $i = 1, ..., n$). 
\item Se crea una matríz $V\in\mathbb{R}^{nxn}$ y en sus columnas se colocan los autovectores $v_i$ de $\^{M}$ en el orden que fueron apareciendo (el método de la potencia con deflación devuelve los autovectores ordenados por módulo y sus autovectores correspondientes). Se guardan aparte los $n$ autovalores $\lambda_i$,  $i = 1, ..., n$.
\item Se calcúla $U=X^t*V$, en cada columna $U$ tendrá $u_i = X^t*v_i$ para $i = 1, ..., n$.
\item Recorriendo las columnas de $U$ se recuperan los autovectores $u_i$ de $M$.

\end{enumerate}\tabularnewline


\subsection{Validación cruzada}
Dado que necesitamos conocer previamente a qué persona corresponde una imagen para
poder estimar la correctitud de la clasificación, una alternativa es particionar la base de
entrenamiento en dos, utilizando una parte de ella en forma completa para el entrenamiento
y la restante como test, pudiendo ası́ corroborar la clasificación realizada, al contar con el
etiquetado del entrenamiento. Sin embargo, realizar toda la experimentación sobre una única
partición de la base podrı́a resultar en una incorrecta estimación de parámetros, dando lugar
al conocido problema de overfitting.\\
Por lo tanto, se estudiará la técnica de $cross validation$, en particular el $K-fold cross
validation 1$ , para realizar una estimación de los parámetros de los métodos que resulte estadı́sticamente más robusta.\\


La validación cruzada \textit{K-fold} consiste en particionar la base de entrenamiento en $K$ partes del mismo tamaño. Luego se realiza $K$ iteraciones, cada una de ellas reteniendo uno de los conjuntos para validación y utilizando los restantes $K - 1s$ para entrenamiento. Este método usualmente permite tomar las particiones sin cuidado alguno, pero en nuestro caso de uso tal cosa no es conveniente. Esto se debe a que en nuestra base de entrenamiento cada persona está representada por diez imágenes, con lo cual dividir las muestras de forma aleatoria puede desbalancear que tan representadas estan algunas personas en el training set. Esto puede significar que el algoritmo se entrena poco para algunas personas y mucho para otras. Yendo más lejos, este desbalance en el train set siempre tiene su contraparte en el test set, impactando de forma negativa las métricas.
Para resolver este problema, se propuso el uso de un k-fold que respete las proporciones de imágenes de cada persona. Como el test set siempre tiene que tener la misma cantidad de fotos para cada persona. Como nuestro dataset cuenta con cuarenta y un personas diferentes, la cantidad de rows del test set será un multiplo de ese número. Esto implica que k será multiplo de diez. Se realizó un shuffle de las imágenes y se eligieron los folds respetando las proporciones mencionadas.



\newpage
\section{Resultados}

Uno de los objetivos de la experimentación de este trabajo era encontrar los mejores parámetros para los métodos, es decir, los parámetros para los cuales los algoritmos proporcionaban mejores resultados. Para determinar la calidad de los resultados obtenidos (cuáles eran los mejores) se tuvo en cuenta distintas métricas, que ayudaron a determinar esto mismo:

\begin{itemize}
\item Accuracy
\item F1-Score
\end{itemize}

Los resultados obtenidos fueron analizados en términos de estas métricas aplicando validación cruzada \textit{K-fold}, variando el $K$ sobre la base de entrenamiento.

Los parámetros $k$ (cantidad de vecinos en \textit{kNN}, no confundir con $K$ de \textit{K-fold}) y $\alpha$ (dimensión a la cual se reduce cada imagen con \textit{PCA}) fueron variándose como se expondrá en las páginas siguientes.

Además de las métricas mencionadas, se tuvo en cuenta el tiempo de computo para las distintas entradas.

\subsection{Tiempo de ejecución kNN}

Dado que el costo temporal de kNN no depende del parámetro k, lo que nos interesa es entender el costo fijo por predicción. En un dataset de 205 imágenes reducidas, el costo promedio de clasificación de una imágen fue de 0.00042. Comparamos ese resultado con el costo promedio para la misma cantidad de imágenes de tamaño completo y el resultado fue 0.11738, varios órdenes de magnitud mayor. 

\subsection{Tiempo de ejecución PCA+kNN}

Para evaluar el costo temporal de las predicciones de kNN al utilizar PCA, es necesario tener en cuenta el costo de la transformación característica (tc), que se realiza a cada imágen a predecir para poder compararla con las imágenes de entrenamiento transformadas gracias al PCA. Otro factor a tener en cuenta es que el $\alpha$ de PCA define la dimensión de las imágenes que compara el kNN, esto impacta directamente a la cantidad de operaciones necesarias y consecuentemente al tiempo de cómputo.

\begin{figure}[H]
	\begin{center}
      \includegraphics[width=0.4\columnwidth]{imagenes/charuli-des/pca_knn_cuantitative_reduced.png}
      \caption{Costo temporal de tc + predicción de kNN}
      \end{center}
\end{figure}

\subsection{Tiempo de ejecución PCA}

Con el objetivo de entender el costo de calcular una matriz de covarianzas para una muestra X, probamos con diferentes cantidades de imagenes reducidas y con diferentes valores del alfa de PCA. 

Para visualizar los resultados de los experimentos para cada X, se creó un gráfico donde la variable independiente es el alpha y la variable dependiente es el tiempo de cómputo. 

En los gráficos Observamos que el tiempo de cómputo se comporta de forma lineal en ralción al alpha para distintos segmentos de alpha, pero la pendiente es discontinua. No se pudo encontrar una causa evidente a esta falta de continuidad.//
Notar que el método descrípto en la sección $Desarrollo$ para obtener la matríz de covarianzas $M$ a partír de $\^{M}$, no pudo ser implementado completamente. Se estima que éste método reduciría significativamente el cálculo de la matríz de covarianza M (que es lo que más tiempo consume de todo el programa). 

\begin{figure}[H]
	\begin{center}
      \includegraphics[width=0.4\columnwidth]{imagenes/charuli-des/pca_cuantitative_redImg_3_rows.png}
      \caption{Costo temporal de PCA para tres imágenes reducidas}
      \end{center}
\end{figure}
\begin{figure}
	\begin{center}
    	\includegraphics[width=0.4\columnwidth]{imagenes/charuli-des/pca_cuantitative_redImg_83_rows.png}
     \caption{Costo temporal de PCA para ochenta y tres imágenes reducidas}
     \end{center}
\end{figure}

\subsection{K de kNN}
 
Utilizamos las imagenes de tamaño completo, sin transformaciones. Llamamos k a la constante que define cuantos vecinos se tienen en cuenta a la hora de decidir la categoría de la muestra a clasificar. Los valores probados para k fueron 1, 3, 8, 10, 15 y 25. Se observó consistentemente que todas las métricas decrecen a medida que se incrementa el valor de k. Siendo $k = 1$ el valor óptimo.
 
\begin{figure}[H]
    \begin{center}
      \includegraphics[width=0.6\columnwidth]{imagenes/charuli-des/k_vs_metricas.png}
      \caption{Metricas para distintos valores de k}
    \end{center}
\end{figure}
 
La diferencia en la calidad de las predicciones tambien puede observarse  visualizando matrices de confusión.
 
\begin{figure}[H]
    \begin{center}
      \includegraphics[width=0.6\columnwidth]{imagenes/charuli-des/Confusion_matrix_for_k_1.png}
      \caption{Metricas para distintos valores de k}
    \end{center}
\end{figure}
 
\begin{figure}[H]
    \begin{center}
      \includegraphics[width=0.6\columnwidth]{imagenes/charuli-des/Confusion_matrix_for_k_25.png}
      \caption{Metricas para distintos valores de k}
    \end{center}
\end{figure}
 
El mejor puntaje obtenido utilizando kNN con k=1 fue accuracy: 0.978049  recall: 0.978049  precision: 0.978049  F1: 0.978049
 
\subsection{Calidad al usar PCA }
 
Al utilizar PCA y kNN tenemos dos parámetros con los cuales experimentar. Se probaron combinaciones entre los k de kNN 1, 3, 8, 10, 15, 25 y los alpha de PCA 3, 10, 25, 40, 60.
 
Para estudiar el comportamiento de K se realizo la operación groupby de pandas DataFrames (escencialmente lo mismo que un groupby de SQL) logrando que para cada valor diferente de k de promedien los valores de cada métrica independientemente del valor de alpha.  Puede observarse al igual que el uso kNN sin pca, los mejores resultados se obtienen mirando un solo vecino.
 
 
 
\begin{figure}[H]
    \begin{center}
      \includegraphics[width=0.6\columnwidth]{imagenes/charuli-des/pca_knn_k_vs_metricas.png}
      \caption{k vs metricas, con alpha promediado}
    \end{center}
\end{figure}
 
Puede observarse la diferencia entre dos matrices de confusión.
 
\begin{figure}[H]
    \begin{center}
      \includegraphics[width=0.6\columnwidth]{imagenes/charuli-des/pca_knn_confusion_matrix_for_k_1.png}
      \caption{Matriz de confusion para k=1, alpha promediado}
    \end{center}
\end{figure}
 
\begin{figure}[H]
    \begin{center}
      \includegraphics[width=0.6\columnwidth]{imagenes/charuli-des/pca_knn_confusion_matrix_for_k_25.png}
      \caption{Matriz de confusion para k=25, alpha promediado}
    \end{center}
\end{figure}
 
 
Para estudiar el comportamiento de alpha se realizó el mismo procedimiento invirtiendo los roles de ambos parámetros. Se puede observar que las métricas mejoran a medida que aumenta alpha,  pero cada vez menos.
\begin{figure}[H]
    \begin{center}
      \includegraphics[width=0.6\columnwidth]{imagenes/charuli-des/pca_knn_alpha_vs_metricas.png}
      \caption{alpha vs metricas, con k promediado}
    \end{center}
\end{figure}
 
Puede observarse la diferencia entre dos matrices de confusión.
 
\begin{figure}[H]
    \begin{center}
      \includegraphics[width=0.6\columnwidth]{imagenes/charuli-des/pca_knn_confusion_matrix_for_alpha_3.png}
      \caption{Matriz de confusion para alpha=3, k promediado}
    \end{center}
\end{figure}
 
\begin{figure}[H]
    \begin{center}
      \includegraphics[width=0.6\columnwidth]{imagenes/charuli-des/pca_knn_confusion_matrix_for_alpha_60.png}
      \caption{Matriz de confusion para alpha=60, k promediado}
    \end{center}
\end{figure}
 
 
Se pueden visualizar ambos efectos utilizando un gráfico de dispersión, en el cual los mejores puntajes son representados por puntos fucsias, cercanos a uno y los peores representados por puntos celestes, cercanos a cero.
\begin{figure}[H]
    \begin{center}
      \includegraphics[width=0.6\columnwidth]{imagenes/charuli-des/k_and_alpha_vs_accuracy.png}
      \caption{alpha y k vs metricas}
    \end{center}
\end{figure}
 
 
El mejor puntaje obtenido utilizando kNN y PCA con k=1 y alpha=60 fue accuracy: 0.978049  recall: 0.978049  precision: 0.979675  F1: 0.978853


\newpage
\input{conclusion}

\newpage
\input{apendice}

% compilar 2 veces para actualizar las referencias


\end{document}