\section{Introducción}

Este trabajo práctico tiene como objetivo el desarrollo y estudio de una herramienta que permita reconocer rostros en imágenes. El algoritmo capaz de llevar esto a cabo es uno de clasificación supervisado que fue entrenado con un lote de fotografías conocido, de modo que le sea posible reconocer otras fotografías de esos rostros aprendidos que no se encuentren en la base de datos de entrenamiento.

Las imágenes de las que disponemos corresponden a cuarenta y un personas, habiendo diez imágenes diferentes por persona.

Si la imagen que entra como parámetro, es de una persona que no pertenece al grupo con que se entrenó al algorítmo, entonces éste indicará a cual de los cuarenta y un se parece más.

\subsection{Evaluación}

Para el estudio de esta herramienta es necesaria la evaluación de los métodos y la correcta elección de sus parámetros. Una forma de evaluación es la estimación de la correctitud de la clasificación, para lo cual es necesita conocer previamente a qué persona corresponde cada imagen. La forma de realizar esto es particionar la base de entrenamiento en dos, utilizando una parte de ella en forma completa para el entrenamiento y la restante como test, pudiendo así corroborar la clasificación realizada, al contar con el etiquetado del entrenamiento.

Sin embargo, realizar toda la experimentación sobre una única partición de la base podría resultar en una incorrecta estimación de parámetros, por ejemplo, podría dar overfitting. Por lo tanto, se implementó la técnica de \textit{K-fold cross validation} que resulta estadísticamente más robusta.

El resultado del algoritmo final fue medido con distintas métricas (Accuracy, Curvas de precisión/recall).

% ____________A CORREGIR_______________

